%
\documentclass{article}

\usepackage[backend=bibtex,
style=numeric
%style=alphabetic
%style=reading
]{biblatex}
\addbibresource{references}

\title{Problem Statement}

\author{
    Greg M. Crist, Jr.
    \and
    Padraic Hagerty
    \and
    Sean Kennedy
    \and
    Miad Rouhani
}

\begin{document}

\maketitle

\section{Introduction}
This is the introduction text. \cite{segal}

\section{}

Vision-based technology is becoming more popular, and almost part of our everyday life as our society is becoming more dependent on these systems for various applications including remote surveillance, medical, and entertainment purposes. As 2D vision technology is starting to lose popularity, 3D imaging is becoming more prevalent, and therefore replacing 2D imaging techniques. This is in part due to recent technological advances. Stereoscopic 3D imaging is not a new concept, it has been a topic of research in the past few decades; however, it was not well received by people due to a lack of quality. This is because software and hardware resources were limited, and this was an obstacle for engineers. Recently, stereoscopic imaging has been gaining ground since hardware performance and memory usage are not as big of an issue nowadays. With the hardware and software resources available for engineers, and a lack of interest by the public for 2D imaging methods, the electronics industry is relying on 3D stereoscopic vision, and therefore more and more products are launched into the market that depend on stereoscopic vision.

Stereoscopic depth and vision are affected both to a great extent by the binocular matching, which is “process that finds which parts of the left and right eye’s images correspond to the same source in the visual scene”. Another factor which has an influences on the depth of perception is binocular disparity.  The concept of binocular disparity refers to the slight difference in position of a visual image as seen by the right and left eye. This is because the right and left eye view objects from different angles, and thus would cause a little visual disparity. Stereoscopic depth perception works when each eye takes the visual information that it receives, and sends it to the brain to be united into one picture, where the slight difference of the images will add perception of depth. This is significant because the added perception is what which will create a 3D image.


\printbibliography

\end{document}

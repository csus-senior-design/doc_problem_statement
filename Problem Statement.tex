\documentclass[12pt, journal]{IEEEtran}
%
% If IEEEtran.cls has not been installed into the LaTeX system files,
% manually specify the path to it like:
% \documentclass[journal]{../sty/IEEEtran}

% Use external references file
\usepackage[backend=bibtex,
style=numeric
%style=alphabetic
%style=reading
]{biblatex}
\addbibresource{references} % references.bib


% Some very useful LaTeX packages include:
% (uncomment the ones you want to load)


% *** MISC UTILITY PACKAGES ***
%
%\usepackage{ifpdf}
% Heiko Oberdiek's ifpdf.sty is very useful if you need conditional
% compilation based on whether the output is pdf or dvi.
% usage:
% \ifpdf
%   % pdf code
% \else
%   % dvi code
% \fi
% The latest version of ifpdf.sty can be obtained from:
% http://www.ctan.org/tex-archive/macros/latex/contrib/oberdiek/
% Also, note that IEEEtran.cls V1.7 and later provides a builtin
% \ifCLASSINFOpdf conditional that works the same way.
% When switching from latex to pdflatex and vice-versa, the compiler may
% have to be run twice to clear warning/error messages.






% *** CITATION PACKAGES ***
%
%\usepackage{cite}
% cite.sty was written by Donald Arseneau
% V1.6 and later of IEEEtran pre-defines the format of the cite.sty package
% \cite{} output to follow that of IEEE. Loading the cite package will
% result in citation numbers being automatically sorted and properly
% "compressed/ranged". e.g., [1], [9], [2], [7], [5], [6] without using
% cite.sty will become [1], [2], [5]--[7], [9] using cite.sty. cite.sty's
% \cite will automatically add leading space, if needed. Use cite.sty's
% noadjust option (cite.sty V3.8 and later) if you want to turn this off
% such as if a citation ever needs to be enclosed in parenthesis.
% cite.sty is already installed on most LaTeX systems. Be sure and use
% version 5.0 (2009-03-20) and later if using hyperref.sty.
% The latest version can be obtained at:
% http://www.ctan.org/tex-archive/macros/latex/contrib/cite/
% The documentation is contained in the cite.sty file itself.






% *** GRAPHICS RELATED PACKAGES ***
%
\ifCLASSINFOpdf
  % \usepackage[pdftex]{graphicx}
  % declare the path(s) where your graphic files are
  % \graphicspath{{../pdf/}{../jpeg/}}
  % and their extensions so you won't have to specify these with
  % every instance of \includegraphics
  % \DeclareGraphicsExtensions{.pdf,.jpeg,.png}
\else
  % or other class option (dvipsone, dvipdf, if not using dvips). graphicx
  % will default to the driver specified in the system graphics.cfg if no
  % driver is specified.
  % \usepackage[dvips]{graphicx}
  % declare the path(s) where your graphic files are
  % \graphicspath{{../eps/}}
  % and their extensions so you won't have to specify these with
  % every instance of \includegraphics
  % \DeclareGraphicsExtensions{.eps}
\fi
% graphicx was written by David Carlisle and Sebastian Rahtz. It is
% required if you want graphics, photos, etc. graphicx.sty is already
% installed on most LaTeX systems. The latest version and documentation
% can be obtained at: 
% http://www.ctan.org/tex-archive/macros/latex/required/graphics/
% Another good source of documentation is "Using Imported Graphics in
% LaTeX2e" by Keith Reckdahl which can be found at:
% http://www.ctan.org/tex-archive/info/epslatex/
%
% latex, and pdflatex in dvi mode, support graphics in encapsulated
% postscript (.eps) format. pdflatex in pdf mode supports graphics
% in .pdf, .jpeg, .png and .mps (metapost) formats. Users should ensure
% that all non-photo figures use a vector format (.eps, .pdf, .mps) and
% not a bitmapped formats (.jpeg, .png). IEEE frowns on bitmapped formats
% which can result in "jaggedy"/blurry rendering of lines and letters as
% well as large increases in file sizes.
%
% You can find documentation about the pdfTeX application at:
% http://www.tug.org/applications/pdftex





% *** MATH PACKAGES ***
%
%\usepackage[cmex10]{amsmath}
% A popular package from the American Mathematical Society that provides
% many useful and powerful commands for dealing with mathematics. If using
% it, be sure to load this package with the cmex10 option to ensure that
% only type 1 fonts will utilized at all point sizes. Without this option,
% it is possible that some math symbols, particularly those within
% footnotes, will be rendered in bitmap form which will result in a
% document that can not be IEEE Xplore compliant!
%
% Also, note that the amsmath package sets \interdisplaylinepenalty to 10000
% thus preventing page breaks from occurring within multiline equations. Use:
%\interdisplaylinepenalty=2500
% after loading amsmath to restore such page breaks as IEEEtran.cls normally
% does. amsmath.sty is already installed on most LaTeX systems. The latest
% version and documentation can be obtained at:
% http://www.ctan.org/tex-archive/macros/latex/required/amslatex/math/





% *** SPECIALIZED LIST PACKAGES ***
%
%\usepackage{algorithmic}
% algorithmic.sty was written by Peter Williams and Rogerio Brito.
% This package provides an algorithmic environment fo describing algorithms.
% You can use the algorithmic environment in-text or within a figure
% environment to provide for a floating algorithm. Do NOT use the algorithm
% floating environment provided by algorithm.sty (by the same authors) or
% algorithm2e.sty (by Christophe Fiorio) as IEEE does not use dedicated
% algorithm float types and packages that provide these will not provide
% correct IEEE style captions. The latest version and documentation of
% algorithmic.sty can be obtained at:
% http://www.ctan.org/tex-archive/macros/latex/contrib/algorithms/
% There is also a support site at:
% http://algorithms.berlios.de/index.html
% Also of interest may be the (relatively newer and more customizable)
% algorithmicx.sty package by Szasz Janos:
% http://www.ctan.org/tex-archive/macros/latex/contrib/algorithmicx/




% *** ALIGNMENT PACKAGES ***
%
%\usepackage{array}
% Frank Mittelbach's and David Carlisle's array.sty patches and improves
% the standard LaTeX2e array and tabular environments to provide better
% appearance and additional user controls. As the default LaTeX2e table
% generation code is lacking to the point of almost being broken with
% respect to the quality of the end results, all users are strongly
% advised to use an enhanced (at the very least that provided by array.sty)
% set of table tools. array.sty is already installed on most systems. The
% latest version and documentation can be obtained at:
% http://www.ctan.org/tex-archive/macros/latex/required/tools/


% IEEEtran contains the IEEEeqnarray family of commands that can be used to
% generate multiline equations as well as matrices, tables, etc., of high
% quality.




% *** SUBFIGURE PACKAGES ***
%\ifCLASSOPTIONcompsoc
%  \usepackage[caption=false,font=normalsize,labelfont=sf,textfont=sf]{subfig}
%\else
%  \usepackage[caption=false,font=footnotesize]{subfig}
%\fi
% subfig.sty, written by Steven Douglas Cochran, is the modern replacement
% for subfigure.sty, the latter of which is no longer maintained and is
% incompatible with some LaTeX packages including fixltx2e. However,
% subfig.sty requires and automatically loads Axel Sommerfeldt's caption.sty
% which will override IEEEtran.cls' handling of captions and this will result
% in non-IEEE style figure/table captions. To prevent this problem, be sure
% and invoke subfig.sty's "caption=false" package option (available since
% subfig.sty version 1.3, 2005/06/28) as this is will preserve IEEEtran.cls
% handling of captions.
% Note that the Computer Society format requires a larger sans serif font
% than the serif footnote size font used in traditional IEEE formatting
% and thus the need to invoke different subfig.sty package options depending
% on whether compsoc mode has been enabled.
%
% The latest version and documentation of subfig.sty can be obtained at:
% http://www.ctan.org/tex-archive/macros/latex/contrib/subfig/




% *** FLOAT PACKAGES ***
%
%\usepackage{fixltx2e}
% fixltx2e, the successor to the earlier fix2col.sty, was written by
% Frank Mittelbach and David Carlisle. This package corrects a few problems
% in the LaTeX2e kernel, the most notable of which is that in current
% LaTeX2e releases, the ordering of single and double column floats is not
% guaranteed to be preserved. Thus, an unpatched LaTeX2e can allow a
% single column figure to be placed prior to an earlier double column
% figure. The latest version and documentation can be found at:
% http://www.ctan.org/tex-archive/macros/latex/base/


%\usepackage{stfloats}
% stfloats.sty was written by Sigitas Tolusis. This package gives LaTeX2e
% the ability to do double column floats at the bottom of the page as well
% as the top. (e.g., "\begin{figure*}[!b]" is not normally possible in
% LaTeX2e). It also provides a command:
%\fnbelowfloat
% to enable the placement of footnotes below bottom floats (the standard
% LaTeX2e kernel puts them above bottom floats). This is an invasive package
% which rewrites many portions of the LaTeX2e float routines. It may not work
% with other packages that modify the LaTeX2e float routines. The latest
% version and documentation can be obtained at:
% http://www.ctan.org/tex-archive/macros/latex/contrib/sttools/
% Do not use the stfloats baselinefloat ability as IEEE does not allow
% \baselineskip to stretch. Authors submitting work to the IEEE should note
% that IEEE rarely uses double column equations and that authors should try
% to avoid such use. Do not be tempted to use the cuted.sty or midfloat.sty
% packages (also by Sigitas Tolusis) as IEEE does not format its papers in
% such ways.
% Do not attempt to use stfloats with fixltx2e as they are incompatible.
% Instead, use Morten Hogholm'a dblfloatfix which combines the features
% of both fixltx2e and stfloats:
%
% \usepackage{dblfloatfix}
% The latest version can be found at:
% http://www.ctan.org/tex-archive/macros/latex/contrib/dblfloatfix/




%\ifCLASSOPTIONcaptionsoff
%  \usepackage[nomarkers]{endfloat}
% \let\MYoriglatexcaption\caption
% \renewcommand{\caption}[2][\relax]{\MYoriglatexcaption[#2]{#2}}
%\fi
% endfloat.sty was written by James Darrell McCauley, Jeff Goldberg and 
% Axel Sommerfeldt. This package may be useful when used in conjunction with 
% IEEEtran.cls'  captionsoff option. Some IEEE journals/societies require that
% submissions have lists of figures/tables at the end of the paper and that
% figures/tables without any captions are placed on a page by themselves at
% the end of the document. If needed, the draftcls IEEEtran class option or
% \CLASSINPUTbaselinestretch interface can be used to increase the line
% spacing as well. Be sure and use the nomarkers option of endfloat to
% prevent endfloat from "marking" where the figures would have been placed
% in the text. The two hack lines of code above are a slight modification of
% that suggested by in the endfloat docs (section 8.4.1) to ensure that
% the full captions always appear in the list of figures/tables - even if
% the user used the short optional argument of \caption[]{}.
% IEEE papers do not typically make use of \caption[]'s optional argument,
% so this should not be an issue. A similar trick can be used to disable
% captions of packages such as subfig.sty that lack options to turn off
% the subcaptions:
% For subfig.sty:
% \let\MYorigsubfloat\subfloat
% \renewcommand{\subfloat}[2][\relax]{\MYorigsubfloat[]{#2}}
% However, the above trick will not work if both optional arguments of
% the \subfloat command are used. Furthermore, there needs to be a
% description of each subfigure *somewhere* and endfloat does not add
% subfigure captions to its list of figures. Thus, the best approach is to
% avoid the use of subfigure captions (many IEEE journals avoid them anyway)
% and instead reference/explain all the subfigures within the main caption.
% The latest version of endfloat.sty and its documentation can obtained at:
% http://www.ctan.org/tex-archive/macros/latex/contrib/endfloat/
%
% The IEEEtran \ifCLASSOPTIONcaptionsoff conditional can also be used
% later in the document, say, to conditionally put the References on a 
% page by themselves.




% *** PDF, URL AND HYPERLINK PACKAGES ***
%
%\usepackage{url}
% url.sty was written by Donald Arseneau. It provides better support for
% handling and breaking URLs. url.sty is already installed on most LaTeX
% systems. The latest version and documentation can be obtained at:
% http://www.ctan.org/tex-archive/macros/latex/contrib/url/
% Basically, \url{my_url_here}.




% *** GLOSSARY PACKAGE ***
%
\usepackage[acronym, nonumberlist]{glossaries}

% This section defines "indexspace", which is used in IEEEtran.cls but it
% is not defined there.
\makeatletter
\newcommand\indexspace{\par \vskip 10\p@ \@plus5\p@ \@minus3\p@\relax}
\makeatother

\makeglossaries

%\setlength{\glsdescwidth}{0.1\linewidth}

\newacronym{2d}{2D}{2-dimensional}
\newacronym{3d}{3D}{3-dimensional}
\newacronym{uav}{UAV}{Unmanned Aerial Vehicle}
\newacronym{gps}{GPS}{Global Positioning System}
\newacronym{imu}{IMU}{Inertial Measurement Unit}
\newacronym{vins}{V-INS}{Vision aided Inertial Navigation System}

\newglossaryentry{stereoscopic}{
    name=stereoscopic,
    description={relating to or denoting a process by which two photographs of the same object taken at slightly different angles are viewed together, creating an impression of depth and solidity}
}
\newglossaryentry{monoscopic}{
    name=monoscopic,
    description={relating to or denoting a process by which a single photograph of an object is used for imaging}
}


% *** Do not adjust lengths that control margins, column widths, etc. ***
% *** Do not use packages that alter fonts (such as pslatex).         ***
% There should be no need to do such things with IEEEtran.cls V1.6 and later.
% (Unless specifically asked to do so by the journal or conference you plan
% to submit to, of course. )


% correct bad hyphenation here
\hyphenation{op-tical net-works semi-conduc-tor}


\begin{document}
%
% paper title
% Titles are generally capitalized except for words such as a, an, and, as,
% at, but, by, for, in, nor, of, on, or, the, to and up, which are usually
% not capitalized unless they are the first or last word of the title.
% Linebreaks \\ can be used within to get better formatting as desired.
% Do not put math or special symbols in the title.
\title{Limitations of Monoscopic Vision Systems in Autonomous and Remotely Operated Applications}

%
%
% author names and IEEE memberships
% note positions of commas and nonbreaking spaces ( ~ ) LaTeX will not break
% a structure at a ~ so this keeps an author's name from being broken across
% two lines.
% use \thanks{} to gain access to the first footnote area
% a separate \thanks must be used for each paragraph as LaTeX2e's \thanks
% was not built to handle multiple paragraphs
%

\author{Padraic~Hagerty,~\IEEEmembership{Student,~CSU~Sacramento,}
        Greg~M.~Crist,~Jr.,~\IEEEmembership{Student,~CSU~Sacramento,}
        Sean~Kennedy,~\IEEEmembership{Student,~CSU~Sacramento,}
        and~Miad~Rouhani,~\IEEEmembership{Student,~CSU~Sacramento,}% <-this % stops a space
%\thanks{M. Shell is with the Department
%of Electrical and Computer Engineering, Georgia Institute of Technology, Atlanta,
%GA, 30332 USA e-mail: (see http://www.michaelshell.org/contact.html).}% <-this % stops a space
%\thanks{J. Doe and J. Doe are with Anonymous University.}% <-this % stops a space
%\thanks{Manuscript received April 19, 2005; revised September 17, 2014.}
}

% note the % following the last \IEEEmembership and also \thanks - 
% these prevent an unwanted space from occurring between the last author name
% and the end of the author line. i.e., if you had this:
% 
% \author{....lastname \thanks{...} \thanks{...} }
%                     ^------------^------------^----Do not want these spaces!
%
% a space would be appended to the last name and could cause every name on that
% line to be shifted left slightly. This is one of those "LaTeX things". For
% instance, "\textbf{A} \textbf{B}" will typeset as "A B" not "AB". To get
% "AB" then you have to do: "\textbf{A}\textbf{B}"
% \thanks is no different in this regard, so shield the last } of each \thanks
% that ends a line with a % and do not let a space in before the next \thanks.
% Spaces after \IEEEmembership other than the last one are OK (and needed) as
% you are supposed to have spaces between the names. For what it is worth,
% this is a minor point as most people would not even notice if the said evil
% space somehow managed to creep in.



% The paper headers
\markboth{CSU Sacramento, EEE/CpE Senior Design Project, Team Honeybadger, Spring 2015}%
{CSU Sacramento, EEE/CpE Senior Design Project, Team Honeybadger, Spring 2015}
% The only time the second header will appear is for the odd numbered pages
% after the title page when using the twoside option.
% 
% *** Note that you probably will NOT want to include the author's ***
% *** name in the headers of peer review papers.                   ***
% You can use \ifCLASSOPTIONpeerreview for conditional compilation here if
% you desire.

% If you want to put a publisher's ID mark on the page you can do it like
% this:
%\IEEEpubid{0000--0000/00\$00.00~\copyright~2014 IEEE}
% Remember, if you use this you must call \IEEEpubidadjcol in the second
% column for its text to clear the IEEEpubid mark.

% use for special paper notices
%\IEEEspecialpapernotice{(Invited Paper)}


% make the title area
\maketitle

% As a general rule, do not put math, special symbols or citations
% in the abstract or keywords.
\begin{abstract}

This report discusses the limitations of monoscopic display systems used in autonomous and remotely operated applications, along with the impact of these limitations, and how they affect a spectrum of different technology fields. The use of stereoscopic vision is suggested as a solution to these limitations of monoscopic display systems, along with a proposal to implement a steroscopic system to counter the limitation of monoscopic vision.
\end{abstract}

% Note that keywords are not normally used for peerreview papers.
\begin{IEEEkeywords}
stereoscopic image capture, stereoscopic displays, stereoscopic vision, display technology, machine vision, machine assisted vision, monoscopic vision, 2D, 3D, robotics
\end{IEEEkeywords}



% For peer review papers, you can put extra information on the cover
% page as needed:
% \ifCLASSOPTIONpeerreview
% \begin{center} \bfseries EDICS Category: 3-BBND \end{center}
% \fi
%
% For peerreview papers, this IEEEtran command inserts a page break and
% creates the second title. It will be ignored for other modes.
\IEEEpeerreviewmaketitle



%\section{Introduction}
% The very first letter is a 2 line initial drop letter followed
% by the rest of the first word in caps.
% 
% form to use if the first word consists of a single letter:
% \IEEEPARstart{A}{demo} file is ....
% 
% form to use if you need the single drop letter followed by
% normal text (unknown if ever used by IEEE):
% \IEEEPARstart{A}{}demo file is ....
% 
% Some journals put the first two words in caps:
% \IEEEPARstart{T}{his demo} file is ....
% 
% Here we have the typical use of a "T" for an initial drop letter
% and "HIS" in caps to complete the first word.
%\hfill mds
%\hfill September 17, 2014

%\subsection{Subsection Heading Here}
%Subsection text here.

% needed in second column of first page if using \IEEEpubid
%\IEEEpubidadjcol

%\subsubsection{Subsubsection Heading Here}
%Subsubsection text here.


% An example of a floating figure using the graphicx package.
% Note that \label must occur AFTER (or within) \caption.
% For figures, \caption should occur after the \includegraphics.
% Note that IEEEtran v1.7 and later has special internal code that
% is designed to preserve the operation of \label within \caption
% even when the captionsoff option is in effect. However, because
% of issues like this, it may be the safest practice to put all your
% \label just after \caption rather than within \caption{}.
%
% Reminder: the "draftcls" or "draftclsnofoot", not "draft", class
% option should be used if it is desired that the figures are to be
% displayed while in draft mode.
%
%\begin{figure}[!t]
%\centering
%\includegraphics[width=2.5in]{myfigure}
% where an .eps filename suffix will be assumed under latex, 
% and a .pdf suffix will be assumed for pdflatex; or what has been declared
% via \DeclareGraphicsExtensions.
%\caption{Simulation results for the network.}
%\label{fig_sim}
%\end{figure}

% Note that IEEE typically puts floats only at the top, even when this
% results in a large percentage of a column being occupied by floats.


% An example of a double column floating figure using two subfigures.
% (The subfig.sty package must be loaded for this to work.)
% The subfigure \label commands are set within each subfloat command,
% and the \label for the overall figure must come after \caption.
% \hfil is used as a separator to get equal spacing.
% Watch out that the combined width of all the subfigures on a 
% line do not exceed the text width or a line break will occur.
%
%\begin{figure*}[!t]
%\centering
%\subfloat[Case I]{\includegraphics[width=2.5in]{box}%
%\label{fig_first_case}}
%\hfil
%\subfloat[Case II]{\includegraphics[width=2.5in]{box}%
%\label{fig_second_case}}
%\caption{Simulation results for the network.}
%\label{fig_sim}
%\end{figure*}
%
% Note that often IEEE papers with subfigures do not employ subfigure
% captions (using the optional argument to \subfloat[]), but instead will
% reference/describe all of them (a), (b), etc., within the main caption.
% Be aware that for subfig.sty to generate the (a), (b), etc., subfigure
% labels, the optional argument to \subfloat must be present. If a
% subcaption is not desired, just leave its contents blank,
% e.g., \subfloat[].


% An example of a floating table. Note that, for IEEE style tables, the
% \caption command should come BEFORE the table and, given that table
% captions serve much like titles, are usually capitalized except for words
% such as a, an, and, as, at, but, by, for, in, nor, of, on, or, the, to
% and up, which are usually not capitalized unless they are the first or
% last word of the caption. Table text will default to \footnotesize as
% IEEE normally uses this smaller font for tables.
% The \label must come after \caption as always.
%
%\begin{table}[!t]
%% increase table row spacing, adjust to taste
%\renewcommand{\arraystretch}{1.3}
% if using array.sty, it might be a good idea to tweak the value of
% \extrarowheight as needed to properly center the text within the cells
%\caption{An Example of a Table}
%\label{table_example}
%\centering
%% Some packages, such as MDW tools, offer better commands for making tables
%% than the plain LaTeX2e tabular which is used here.
%\begin{tabular}{|c||c|}
%\hline
%One & Two\\
%\hline
%Three & Four\\
%\hline
%\end{tabular}
%\end{table}


% Note that the IEEE does not put floats in the very first column
% - or typically anywhere on the first page for that matter. Also,
% in-text middle ("here") positioning is typically not used, but it
% is allowed and encouraged for Computer Society conferences (but
% not Computer Society journals). Most IEEE journals/conferences use
% top floats exclusively. 
% Note that, LaTeX2e, unlike IEEE journals/conferences, places
% footnotes above bottom floats. This can be corrected via the
% \fnbelowfloat command of the stfloats package.



\section{Introduction}
\IEEEPARstart{T}{he} use of machine vision and machine assisted human vision is increasingly utilized in many aspects of everyday life.  From industrial, military, business, medical, and entertainment applications, the integration of machine vision and interaction with physical objects is more common than ever.  More importantly, the integration of machine vision systems with remotely operated machinery, robotics, and vehicles is a crucial innovation that provides safety for people from hazardous environments, the ability to allow a specialist to offer expertise from a distant location, as well as bridge real and virtual environments and facilitate communication and interaction from remote locations.

However, while the integration of machine vision and remote systems operation is possible and occurring today, it is no substitute for first-person operation.  Despite the vast technological innovations in vision capture systems, two-dimensional, or gls{monoscopic} displays are most prevalent, and this type of display introduces errors and limits the abilities of those using such a system.

Due to the limitations of monoscopic vision, the use of stereoscopic vision capture and display is suggested as a solution.  Additionally, the implementation of such as system is proposed to be developed for the EEE/CPE Senior Design Project at California State University, Sacramento for the duration of the Spring 2015 and Fall 2015 Semesters.


\section{Societal Problem}
Most imaging technology used throughout society today is monoscopic, meaning it is a single camera system that provides an image to an end user. Single camera systems have some fundamental limits which constrain the system's functionality. Monoscopic image capture and display systems do not provide depth perception to a human that is using the system, or to a machine that is processing the incoming visual data. This means that the end user of the system, be it man or machine, has to deal with controlling something without access to the third spatial dimension. Obviously this creates a problem for accurately controlling any sort of mechanical system since having three dimensional vision allows the end user to judge depth much more accurately. Another problem posed by monoscopic vision is that the field of view is a much smaller angle than that of a stereoscopic imaging system like human eyes. This can easily be confirmed by comparing the field of view an individual has with both eyes open to the field of view with just one eye open.

Monoscopic imaging systems are on most currently deployed UAVs (Unmanned Aerial Vehicles), remotely operated aerial vehicles, and other remotely operated robotic systems. Due to the limitations of monoscopic images, UAVuse a GPS (Global Positioning System) and IMUs (Inertial Measurement Unit) so the vehicle can fly autonomously \cite{sanfourche}. These sensing systems only allow autonomous flight at middle to high altitudes since they cannot account for obstacles at lower altitudes \cite{sanfourche}. Since monoscopic imaging systems do not provide three dimensional views, both autonomous and remote controlled flight are not possible to do safely at low altitudes. Since autonomous flight with UAVs is so heavily dependent on a GPS, conditions where a GPS is not available pose a serious problem for UAVs \cite{chowdhary}. One of the possible solutions for this problem is to use a V-INS (Vision aided Inertial Navigation System), but these systems suffer from being computationally intensive \cite{chowdhary}. Even if monoscopic imaging could be used to accurately judge relative distances between a UAV and objects near it, the limited field of view caused by a monoscopic system would not allow the UAV to see objects that could still cause collisions with the craft.

Another field that is affected by issues with monoscopic imaging is medicine. Surgeons using a monoscopic system to perform operations on patients are forced to control robotic arms with no sense of depth. This is an application where accuracy is essential for avoiding any collateral damage to the patient. Research has shown significant differences in the performance of surgeons between using a monoscopic system and a stereoscopic system \cite{munz}. Surgeons using the monoscopic system required more movements to complete tasks than the surgeons using the stereoscopic system, and the surgeons' movements using the monoscopic system were less accurate than the movements of the surgeons using the stereoscopic system \cite{munz}.


\section{Impact}
2D imaging technology is becoming less popular, and it is being replaced by stereoscopic imaging technology, which can provide three dimensional images. Stereoscopic vision systems are becoming part of our everyday life, and our society is becoming more dependent on these systems for various applications including remote surveillance, medical, and entertainment purposes. As the 2D vision technology is gradually starting to become obsolete, 3D imaging on the other hand is becoming more prevalent, and therefore replacing 2D imaging techniques.  This is in part due to the technological advances in this decade.  

Although, stereoscopic imaging is not relatively new since it has been a topic of research in the past few decades; however, it was not well received by people due to a lack of quality\cite{bryant}. This is because the software as well as hardware resources were limited, and thus was an obstacle for engineers. Recently, stereoscopic imaging is gaining ground among people, as performance and memory are not as big of an concern nowadays. With the hardware and software resources available for engineers and a lack of interest by the public for 2D and various other methods of imaging, the electronic industry is relying on 3D stereoscopic vision, and therefore more products are launched into the market that depend on stereoscopic vision \cite{bryant}. 2D vision systems on the other hand have drawbacks that can negatively effect various fields and certain applications.

The most profound impact of 2D vision is that it can be inaccurate and cause errors. This is because a lack of depth and reduced object detection can produce incorrect results in performing tasks where full machine vision or machine assisted human vision is used.  Depth and vision are affected both to a great extent by the binocular matching, which is a process that finds which parts of the left and right eye’s images correspond to the same source in the visual scene. Another factor which has an influences on the depth of perception is binocular disparity.  The concept of binocular disparity refers to the slight difference in position of a visual image as seen by the right and left eye \cite{naquet}. The reason is because the right and left eye view objects from different angles, and thus would cause a little visual disparity. Stereoscopic depth perception works when each eye takes the visual information that it receives, and sends it to the brain to be united into one picture, where the slight difference of the images will add the illusion and perception of depth \cite{naquet}. 2D vision lacks both binocular matching and disparity, and thus it has no depth, which may lead to erroneous and undependable results.

Monoscopic vision can be a disadvantage in the field of intelligent systems including robots and remote operated vehicles. Intelligent systems such as robots will require precision and accurate results when capturing data. Robots and intelligent systems capture data and based on the data that is received, then an action of some sort would be taken by the robot depending on the application. However, since the 2D data that will be received may be inaccurate because, by using monocular vision the distance of the object cannot as accurately be found \cite{segal}. This could cause an incorrect course of action to be taken, which can be of great concern when dealing with precision and accuracy. Another issue that occurs from monoscopic vision is that with the lack of depth perception a misrepresentation of a objects depth or width can change how we perceive it and provide a misconception of the objects size or shape of the object \cite{pollock}. The concern of exocentric distance perception in identifying and navigation with a monoscopic camera in medical devices could cause concern in specific medical devices such as laparoscopic devices are used in surgeries to navigate and identify particular areas in the human body \cite{pollock}. 

\section{Solution}
As a result of the limitations of monoscopic vision, researchers have investigated the use of both simulated and true stereoscopic vision as an alternative solution when a vision system is necessary.  Consistently, studies have shown that in a variety of scenarios ranging from complete machine vision to human vision systems, the use of true stereoscopic vision provides better results than when monoscopic vision is utilized.

In an article discussing the tracking of objects in orbit in space, the use of stereoscopic imaging in a machine-vision system yielded faster target tracking and increased robustness to field-of-view limits noise compared with monoscopic vision-based attitude tracking \cite{segal}. Additionally, these improvements had a fringe benefit of reduced fuel consumption in the systems utilizing the stereoscopic vision.  This is specifically attributed to the ability to accurately implement a model of the object being tracked without any ambiguity in the relative range since the angular acceleration can be measured using the stereoscopic vision.

In human vision systems where the use of cameras augments a process or procedure by a person, research has shown that the use of stereoscopic vision improves the performance and results compared to monoscopic vision.  In a clinical study of using cameras to assist in dental implant procedures, the use of stereoscopic vision through independent optical channels improved precision in dental implant procedures compared to non-stereoscopic vision \cite{wanschitz}.  In another study utilizing the Da Vinci robotic assisted surgical system, researchers aimed to objectively assess whether stereoscopic vision improved the performance over monoscopic vision. The results of the study indicated that for all tasks and parameters there were significant improvements \cite{munz}. Additionally, a fewer errors were encountered while using the stereoscopic mode as compared to the standard two-dimensional image \cite{munz}.  Furthermore, in a paper published specifically to address the use of stereoscopic imaging in medical procedures, the concluding remarks indicate that stereoscopic vision offers significant clinical improvements for many aspects of medicine over monoscopic vision \cite{held}.  Specifically, in diagnostics, training, and remote surgery, the article identifies shortened task-completion times and reduced error rates due to enhanced shape perception in the absence of other visual cues.

Based upon the evidence, stereoscopic vision is an improvement over monoscopic vision.  For a system where the display of a remote environment is necessary, the use of a stereoscopic capture and display would be more efficient, accurate, and provide better results than a monoscopic system

\section{Proposal}
To address the limitations of a remotely operated robotic system utilizing a monoscopic display, the team is proposing to design and build a remotely operated robotic manipulation device that integrates stereoscopic vision.

The system that is proposed would consist of a custom hardware design to capture stereoscopic image data through low-cost CMOS cameras and transmit it to a remote head-mount stereoscopic display.  To provide a more natural interaction and response, the head-mount display would integrate motion tracking to reposition the remote cameras.  Additionally, a remotely operated robotic control system would be utilized to demonstrate the effectiveness of the stereoscopic vision system.

If time and resources permitted, additional enhancements to the system could further improve the experience, such as implementing super-resolution image processing to improve the total resolution of the image data, object recognition and tracking, and additional feedback mechanisms.


\section{Conclusion}
While the problems and limitations of monoscopic vision may seem narrow in scope, the implications are felt in a number of fields that affect the safety and health of individuals, operational costs of various industries, and limit the use of technology developed to otherwise facilitate tasks in ways that are not possible without it.

Researched evidence supports the effectiveness of the suggested solution of replacing monoscopic vision with stereoscopic vision, and is believed to be attainable with available technology and tools.  The proposed implementation is reasonable in scope for the time frame spanning the Spring 2015 and Fall 2015 semesters at California State University, Sacramento, and given the abilities of the team members.


% if have a single appendix:
%\appendix[Proof of the Zonklar Equations]
% or
%\appendix  % for no appendix heading
% do not use \section anymore after \appendix, only \section*
% is possibly needed

% use appendices with more than one appendix
% then use \section to start each appendix
% you must declare a \section before using any
% \subsection or using \label (\appendices by itself
% starts a section numbered zero.)
%


%\appendices
%\section{Proof of the First Zonklar Equation}
%Appendix one text goes here.

% you can choose not to have a title for an appendix
% if you want by leaving the argument blank
%\section{}
%Appendix two text goes here.


% use section* for acknowledgment
%\section*{Acknowledgment}


%The authors would like to thank...


% Can use something like this to put references on a page
% by themselves when using endfloat and the captionsoff option.
\ifCLASSOPTIONcaptionsoff
  \newpage
\fi



% trigger a \newpage just before the given reference
% number - used to balance the columns on the last page
% adjust value as needed - may need to be readjusted if
% the document is modified later
%\IEEEtriggeratref{8}
% The "triggered" command can be changed if desired:
%\IEEEtriggercmd{\enlargethispage{-5in}}


% glossary section
\glsaddall
\glossarystyle{altlist}
\printglossaries


% references section

% can use a bibliography generated by BibTeX as a .bbl file
% BibTeX documentation can be easily obtained at:
% http://www.ctan.org/tex-archive/biblio/bibtex/contrib/doc/
% The IEEEtran BibTeX style support page is at:
% http://www.michaelshell.org/tex/ieeetran/bibtex/
%\bibliographystyle{IEEEtran}
% argument is your BibTeX string definitions and bibliography database(s)
%\bibliography{IEEEabrv,../bib/paper}
%
% <OR> manually copy in the resultant .bbl file
% set second argument of \begin to the number of references
% (used to reserve space for the reference number labels box)

%\begin{thebibliography}{1}
%
%\bibitem{IEEEhowto:kopka}
%H.~Kopka and P.~W. Daly, \emph{A Guide to \LaTeX}, 3rd~ed.\hskip 1em plus
%  0.5em minus 0.4em\relax Harlow, England: Addison-Wesley, 1999.
%
%\end{thebibliography}

% Print the bibliography using the external bibtex file (see top of doc)
\printbibliography


% biography section
% 
% If you have an EPS/PDF photo (graphicx package needed) extra braces are
% needed around the contents of the optional argument to biography to prevent
% the LaTeX parser from getting confused when it sees the complicated
% \includegraphics command within an optional argument. (You could create
% your own custom macro containing the \includegraphics command to make things
% simpler here.)
%\begin{IEEEbiography}[{\includegraphics[width=1in,height=1.25in,clip,keepaspectratio]{mshell}}]{Michael Shell}
% or if you just want to reserve a space for a photo:

%\begin{IEEEbiography}{Michael Shell}
%Biography text here.
%\end{IEEEbiography}

% if you will not have a photo at all:
%\begin{IEEEbiographynophoto}{John Doe}
%Biography text here.
%\end{IEEEbiographynophoto}

% insert where needed to balance the two columns on the last page with
% biographies
%\newpage

%\begin{IEEEbiographynophoto}{Jane Doe}
%Biography text here.
%\end{IEEEbiographynophoto}

% You can push biographies down or up by placing
% a \vfill before or after them. The appropriate
% use of \vfill depends on what kind of text is
% on the last page and whether or not the columns
% are being equalized.

%\vfill

% Can be used to pull up biographies so that the bottom of the last one
% is flush with the other column.
%\enlargethispage{-5in}



% that's all folks
\end{document}